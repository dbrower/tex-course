\documentclass{article}

\usepackage{booktabs}
\usepackage{titlesec}
\usepackage{framed}
\usepackage{url}
\usepackage{graphicx}
%\usepackage{showlabels}

\titleformat{\section}[runin]
{\normalfont\bfseries}
{\S\thesection.}{.5em}{}[.]
\titlespacing{\section}
{\parindent}{1.5ex plus .1ex minus .2ex}{10pt}


\begin{document}

\title{Notre Dame \LaTeX\ Thesis Template Guide}
% \author{Don Brower\thanks{email: \texttt{dbrower$@$nd.edu}}}
\date{October 2014}
\maketitle

\begin{itemize}
\item The files are on the graduate school's website at \url{http://graduateschool.nd.edu/resources-for-current-students/dt/dt-resources/}.
In the middle of the page is a view of a Box folder. Choose \textit{LaTeX (nddiss2e)}, and then download the file \textit{StandDistribution.zip}.
Unziping the file creates a directory \textit{StandDistribution} with five files inside:

\begin{itemize}
	\item nddiss2e.bst
	\item nddiss2e.cls
	\item nddiss2e.pdf
	\item nddiss2enoarticletitles.bst
	\item template.tex
\end{itemize}

\item The file \textit{nddiss2e.pdf} explains the options of the thesis class, e.g. how to make draft versions, final versions, the options for the title page, etc.

\item One difficulty of editing a large document is the time spent reprocessing each chapter after changes.
    If it makes sense for your work, I would recommend putting each chapter in its own file, or even its own directory.
    This is how I solved the problem.
    You may find a better solution, or perhaps your editor supports processing files independently.
    For example:

\begin{verbatim}
my-thesis/
    chapter
    Makefile
    don-thesis.bib
    dwip
        dwip-forking.tex
        dwip.tex
        modeltheory.bib
    hyperimaginaries
        hyper-only.tex
        hyper.tex
    introduction.tex
    nddiss2e.bst
    nddiss2e.cls
    preamble.tex
    seq-amalg
        modeltheory.bib
        seq-only.tex
        seq.tex
    thesis.tex
\end{verbatim}

The file \textit{preamble.tex} contains any shared definitions which I intend to use across the chapters.
(e.g. I defined the symbol \texttt{\\sss} which are three large stars so I can see places that I left incomplete.
\verb=\newcommand{\sss}{\ensuremath{\star\star\star}}=)

The main thesis is defined in the \textit{thesis.tex} file.
I set up the preamble, title, abstract, table of contents, and so on.
Then I include each chapter with code similar to the following

\begin{verbatim}
\chapter{DWIP}
\label{ch:dwip}
\input{dwip/dwip}
\end{verbatim}

(N.b. Chapter titles in the thesis should be uppercased.)
Inside each chapter there are two files.
The main content is in one file, e.g. \textit{dwip.tex}.
I have a second file which wraps the content file so it can be processed by itself.

\begin{verbatim}
\documentclass[12pt]{amsart}
\input{../preamble}
\begin{document}
\title{DWIP in Simple Theories}
\maketitle
\input{dwip}
\bibliographystyle{plain}
\bibliography{modeltheory}
\end{document}
\end{verbatim}

\end{itemize}

\end{document}
