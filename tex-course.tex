\documentclass{article}

\usepackage{booktabs}
\usepackage{titlesec}
\usepackage{framed}
\usepackage{hyperref}
\usepackage{graphicx}
%\usepackage{showlabels}
%\usepackage{charter}

\titleformat{\section}[runin]
{\normalfont\bfseries}
{\S\thesection.}{.5em}{}[.]
\titlespacing{\section}
{\parindent}{1.5ex plus .1ex minus .2ex}{10pt}

\titleformat{\subsection}[runin]
{\normalfont\scshape}
{\S\thesubsection.}{.5em}{}[.]
\titlespacing{\subsection}
{\parindent}{1.5ex plus .1ex minus .2ex}{10pt}


\begin{document}

\title{Introduction to \LaTeX\ Workshop}
\author{Don Brower\thanks{email: \texttt{dbrower$@$nd.edu}} \\ Navari Family Center for Digital Scholarship}
\date{2021-11-18}
\maketitle

\section{What is \TeX?}

\TeX\ is a system to produce formatted documents.
While it can produce all kinds of documents, such as Articles, Slides, CVs, Handouts, and Exams,
it is especially useful for items with mathematical formula (but that is not a requirement).

\TeX\ is a mark-up language, akin to HTML.
One writes a text file containing commands that tell \TeX\ how to format the file.
Then one runs \TeX\ on it, and (hopefully) a beautiful PDF document is produced.

Specifically, this workshop is on \LaTeX, which is an extension to \TeX\ that makes it easier to use.
The workshop assumes you have no experience with \LaTeX, but that you do have some computer experience.
The goal is for you to be able to understand the basic \LaTeX\ concepts and terminology so that you can
find and understand other resources and guides for \LaTeX, both online and in books.
A small list of these resources is presented at the end, in \S\ref{sec:further-reading}.

To show off the abilities of this new typesetting software, Donald Knuth, the creator of \TeX,
made a fancy arrangement of the letters: \TeX.
I find this visually annoying, so I will use TeX and LaTeX to refer to the two programs.

\section{History of TeX and LaTeX}

The first version of the TeX system was created in 1978.
It was written by a Donald Knuth, a computer scientist at Stanford University,
to solve the problem of typesetting mathematical articles.
In 1986 Leslie Lamport developed LaTeX as a way to make TeX easier to use.
Nowadays, almost everyone uses LaTeX, and uses TeX and LaTeX to refer to the same
thing.
This course will treat the two terms interchangeably.

\section{Where is \TeX?}

This workshop will use an online version of \LaTeX\ at \url{overleaf.com}.
Notre Dame has a site licence, and if you register using your Notre Dame email address, it will be free.

Using an online service is nice since you will not need to install TeX on your computer (which is possible).
An online service also makes it easier to collaborate on writing documents with others.


\section{Hello World}
Start by making a new project: choose the button \textsf{New Project} and then choose \textsf{Blank Project}.
Type in a name for this project, say ``Latex workshop''.
A new project is created and you should see a document with the following text in it:

\begin{framed}
\begin{verbatim}
\documentclass{article}
\usepackage[utf8]{inputenc}

\title{workshop}
\author{Don Brower}
\date{May 2019}

\begin{document}

\maketitle

\section{Introduction}

\end{document}
\end{verbatim}
\end{framed}

Click the ``Recompile'' button and you should see a page with the your name, today's date, and the word ``Introduction''.

Now type ``hello world'' on the line after the word ``Introduction'', and recompile. You should see your changes appear in the PDF on the right-hand side.

This is a LaTeX file. The commands that start with a backslash, \textbackslash, tell LaTeX that the following word is a formatting command.
The first backslashed word, \verb=\documentclass=, tells LaTeX style package to use, in this case an article.\footnote{Some other formatting packages are \textsf{book}, \textsf{beamer} (to make slide presentations), or \textsf{nddiss2e} to format ND dissertations.}

The pair of commands \verb=\begin{document}= and \verb=\end{document}= enclose the actual text to be typeset.
The text in between \verb=\documentclass= and \verb=\begin{document}= is called the \emph{preamble}.
The preamble is used to load additional packages we wish to use, or to define custom commands for use in our document.

% makes a bunch of files: hello.log, hello.aux, hello.dvi, hello.blg, hello.bbl, hello.pdf, hello.synctex.gz

In addition to the backslash, there are other special characters to be aware of:
\begin{center}
    \% \$ \{ \} \_ \# \& \textasciitilde\ \textgreater\ \textless\
\end{center}
For now know that if you want to typeset one of them you will need to use a special command:
\begin{center}
    \begin{tabular}[]{ll}
        \toprule
        Special Character & Command \\
        \midrule
        \textbackslash & \verb=\textbackslash= \\
        \% & \verb=\%= \\
        \$ & \verb=\$= \\
        \{ & \verb=\{= \\
        \} & \verb=\}= \\
        \_ & \verb=\_= \\
        \# & \verb=\#= \\
        \& & \verb=\&= \\
        \textasciitilde & \verb=\textasciitilde= \\
        \textgreater & \verb=\textgreater= \\
        \textless & \verb=\textless= \\
        \bottomrule
    \end{tabular}
\end{center}

For example to get ``That was \$100.89'', type \verb=That was \$100.89=.


\section{Text Formatting}
We can make text bold and italic with the \verb=\textbf{}= and \verb=\textit{}= commands.

\begin{framed}
\begin{verbatim}
This is \textbf{bold text}, and this is \textit{italic}.
They \textit{can be \textbf{nested}}.
\end{verbatim}
\end{framed}

It should produce something which looks like the following.

\begin{center}
\begin{framed}
This is \textbf{bold text}, and this is \textit{italic}.
They \textit{can be \textbf{nested}}.
\end{framed}
\end{center}

These commands start with a backslash, followed by an opening brace, \{, the text to format, and then a closing brace, \}.
The opening and closing braces need to be matched.
If they are not matched, LaTeX will complain.

Try using the following character styles:

\begin{center}
    \begin{tabular}[]{ll}
        \toprule
        Style & Command \\
        \midrule
Normal Text     & \verb=\textnormal{...}= \\
Emphasis        & \verb=\emph{...}= \\
Roman Font      & \verb=\textrm{...}= \\
Sans Serif Font & \verb=\textsf{...}= \\
Monospaced Font & \verb=\texttt{...}= \\
Italic Shape    & \verb=\textit{...}= \\
Slanted Shape   & \verb=\textsl{...}= \\
Small Capitals  & \verb=\textsc{...}= \\
Uppercase       & \verb=\uppercase{...}= \\
Lowercase       & \verb=\lowercase{...}= \\
Bold Shape      & \verb=\textbf{...}= \\
Medium Weight   & \verb=\textmd{...}= \\
\bottomrule
    \end{tabular}
\end{center}

\section{Spaces, Justification, and Hyphenations}

Most of the time you will use normal spaces in your document.
LaTeX will automatically format your lines, and group them into paragraphs.
It will even hyphenate words to improve the line breaks.

LaTeX will put lines which are next to each other into the same paragraph.
To start a new paragraph, enter a blank line.

\begin{framed}
\begin{verbatim}
This line will be in a paragraph.
And this line will
be in the same paragraph.

This line is in a new paragraph.

And this line is put
in a third paragraph. Along with
this line.
\end{verbatim}
\end{framed}

Produces output like the following:

\begin{framed}
This line will be in a paragraph.
And this line will
be in the same paragraph.

This line is in a new paragraph.

And this line is put
in a third paragraph. Along with
this line.
\end{framed}

By default, paragraphs are fully justified.
LaTeX will break lines in places that gives the best justification, hyphenating words where needed.
Multiple spaces are collapsed into a single space.
Some comments:

\begin{itemize}
    \item Keep LaTeX from spliting a line at a given space by replacing the space with a tilde, \textasciitilde.
        For example, \verb=Mrs.~White= will not insert a line break between \verb=Mrs.= and \verb=White=.\footnote{See \url{https://tex.stackexchange.com/questions/15547/when-should-i-use-non-breaking-space} for more discussion.}

    \item A line break can be forced by ending a line with two backslashes, \textbackslash\textbackslash

    \item If LaTeX gets the hyphenation of a word wrong, you can tell LaTeX the correct hyphenation with \verb=\hyphenation{...}=.
        E.g. \verb=\hyphenation{wysiwyg as-tro-labe}= tells LaTeX to not hyphenate \textit{wysiwyg}, and that \textit{astrolabe} can be split after the \textit{as} or before the \textit{labe}.
        This command will be needed for technical vocabulary that the default rules get wrong.
        (Confession: I have never needed to use this command.)

    \item Force a page break with \verb=\pagebreak=

    \item Make horizontal spaces with the \verb=\hspace{...}= command. For example, this line has \hspace{1in} \verb=\hspace{1in}= 1 inch of space.

    \item A common reason for horizontal space is to leave blank space for worksheets.
        Sometimes a line is desired for those. In LaTeX that line is called a \textit{rule}. E.g. \verb=Name: \rule[-.5ex]{2in}{.1pt}=, Name: \rule[-.5ex]{2in}{.1pt}.
\end{itemize}

\section{Quotes and Comments}

Don't use the usual quote character \verb="= in your files.
Instead use two back-ticks \verb=``= as an open quote and two apostrophes as a close quote:
\verb=``Quote''= to get the output ``Quote''.

There are three kinds of dashes: the hyphen, - ; the en-dash, -- ; and the em-dash, ---.
The hyphen is used when breaking words between lines and when combining watcha-may-call-its.
The en dash, made using \verb=--=, is used to indicate number ranges, e.g. 5--10 business days.
The em dash, made with \verb=---=, is used to indicate changes in thought---or, whatever.

% discuss comments
% TODO: insert cat picture here?
The percent sign, \%, starts a comment inside your source file.
LaTeX will discard the percent sign as well as any remaining
text on the line.   % comments can also end a line
This can be useful to either leave notes to yourself in the
source file, or to temporary remove text without deleting it.

\begin{framed}
\begin{verbatim}
% discuss comments
% TODO: insert cat picture here?
The percent sign, \%, starts a comment inside your source file.
LaTeX will discard the percent sign as well as any remaining
text on the line.   % comments can also end a line
This can be useful to either leave notes to yourself in the
source file, or to temporary remove text without deleting it.\end{verbatim}
\end{framed}

\section{Environments}

We have already seen the environment \verb=document=.
Environments enclose sections of text, and they all begin with the command \verb=\begin{...}=
    and conclude with the command \verb=\end{...}=.
Some other environments are \verb=center=, \verb=quote=, \verb=flushleft=, and \verb=flushright=.

\begin{framed}
    \begin{verbatim}
\begin{center}
    This line will be in its own paragraph, centered.
\end{center}
\begin{flushright}
    This paragraph will be aligned
    against the right margin.
    All the lines will be pushed as far to the right
    as possible.
    Lines will wrap themselves, or we can force \\
    a new line.
\end{flushright}\end{verbatim}
\end{framed}
Produces
\begin{framed}
    \begin{center}
    This line will be in its own paragraph, centered.
    \end{center}
    \begin{flushright}
        This paragraph will be aligned
        against the right margin.
        All the lines will be pushed as far to the right as they can.
        Lines will wrap themselves, or we can force \\
        a new line.
    \end{flushright}
\end{framed}


\section{Lists}

There are three kinds of lists in Latex: Enumerated Lists, Itemized Lists, and Description lists.

\begin{itemize}
    \item An itemized list provides a sequence of bullet points.
    \item Each point may be its own paragraph
    \item the overall list is offset in indented
\end{itemize}
\begin{framed}
    \begin{verbatim}
\begin{itemize}
    \item An itemized list provides a sequence of bullet points.
    \item Each point may be its own paragraph
    \item the overall list is offset in indented
\end{itemize}\end{verbatim}
\end{framed}
Each bullet point is started with an \verb=\item= command.
The other kinds of lists are similar.
\begin{enumerate}
    \item Enumerated lists are numbered, or lettered.
    \item Lists can even be nested.
        \begin{enumerate}
            \item This list appears inside an item for the outer list.
            \item Another item, just to create a second point.
        \end{enumerate}
    \item The last enumerated sentence.
\end{enumerate}
\begin{framed}
\begin{verbatim}\begin{enumerate}
    \item Enumerated lists are numbered, or lettered.
    \item Lists can even be nested.
        \begin{enumerate}
            \item This list appears inside an item for the outer list.
            \item Another item, just to create a second point.
        \end{enumerate}
    \item The last enumerated sentence.
\end{enumerate}\end{verbatim}
\end{framed}
\begin{description}
    \item[Description Lists] A description list consists of pairs
        of terms and descriptions.
    \item[Intention] While this description list is using the
        item headings as on organizational tool, it is intended
        to use them as a way to define a bunch of terms.
    \item[Occasionally] \hfill \\
        You may want the terms to be on seperate lines than the
        terms being defined.
        Then the ``\textbackslash hfill newline'' trick is helpful.
\end{description}
\begin{framed}
\begin{verbatim}
\begin{description}
    \item[Description Lists] A description list consists of pairs
        of terms and descriptions.
    \item[Intention] While this description list is using the
        item headings as on organizational tool, it is intended
        to use them as a way to define a bunch of terms.
    \item[Occasionally] \hfill \\
        You may want the terms to be on seperate lines than the
        terms being defined.
        Then the ``\textbackslash hfill newline'' trick is helpful.
\end{description}\end{verbatim}
\end{framed}

\section{Sections and Table of Contents}

Sections are started by using the \verb=\section{}= command.
The text in the braces is the section title.
Everything following a section command (until the next section command) is considered part of that section.
There are also commands for \verb=\subsection= and \verb=\subsubsection=.

The exact sectioning commands available depends on the \verb=\documentclass= your document is using.
For example, the book type includes \verb=\chapter= commands.

A table of contents is inserted using the command \verb=\tableofcontents=.

\begin{framed}
\begin{verbatim}
\tableofcontents

\section{Introduction}
In this report we revisit the early research done by...

\subsection{Layout of the report}
More text here...

\subsubsection{Points and Counterpoints}
And more...

\subsection{Goals}
And more...
\end{verbatim}
\end{framed}

You will need to run latex \textit{twice} for this to turn out.
The first time latex records the titles of the sections and the second time they will be inserted into the table of contents.

\section{References}
\label{sec:references}

LaTeX makes it easy to generate cross-references between sections.
Most sectioning commands and figure commands can be labeled using a \verb=\label=
directive, which assigns the command an internal name.
To refer to the section/figure/table one then uses a \verb=\ref= command with the same internal name.
For example, the following code will tag the section ``Further Reading'' with the label
\verb=sec:further-reading=.
\begin{framed}\begin{verbatim}
\section{Further Reading}
\label{sec:further-reading}
\end{verbatim}
\end{framed}
To refrer to the section we then use the \verb=\ref= command as follows:
\begin{framed}\begin{verbatim}
See section~\ref{sec:further-reading}.
\end{verbatim}
\end{framed}
The tilde, \verb=~=, inserts a non-breaking space to keep the section number from being separated from the word ``section''.

I usually use prefixes, like \texttt{sec:}, \texttt{eq:}, and \texttt{tbl:},
to remind me whether the label is referring to a section, subsection, table, figure, or equation.
But they are not required by latex.

The \texttt{showlabels} package will display all your labels in the margin next to the item being tagged.
It is very handy when writing initial drafts.
Add the line \verb=\usepackage{showlabels}= to the preamble to use it.



\section{Math}

The ability to typeset math formulae is the main reason many people
choose to use LaTeX at first.
A formula may be typeset \emph{inline}, such
as $x$ and $f(z) = e^z$.
More complex formula can be typeset in \emph{display style}:
\[
    H(a, r) = \sum_{n = 0}^\infty ar^n = \frac{a}{1-r}.
\]
Inline formulae are surrounded by dollar signs, \$, and display formulae
use \verb=\[= and \verb=\]=:
\begin{framed}
\begin{verbatim}
A formula may be typeset \emph{inline}, such
as $x$ and $f(z) = e^z$.
More complex formula can be typeset in \emph{display style}:
\[
    H(a, r) = \sum_{n = 0}^\infty ar^n = \frac{a}{1-r}.
\]\end{verbatim}
\end{framed}

The text between the dollar signs or the \verb=\[= and \verb=\]= is said to
be in \textit{math mode}.
In math mode, text is treated differently than in the usual paragraph mode.
Spaces in math mode are not typeset.
In math mode, one can use carets, \^\ , and underscores, \_\ , for superscripts and subscripts, respectively.
Other commands are only available in math mode, for example \verb=\frac{..}{..}= to make fractions, and
\verb=\sum= for the summation symbol.
You can get Greek letters using \verb=$\alpha, \beta, \gamma, \ldots$=, to give $\alpha, \beta, \gamma,\ldots$

There are, unfortunately, too many math commands to discuss each here.
If you have especially demanding math needs, the \verb=amsmath= package has many specialized environments for mathematics.

\section{Pictures and Packages}

To include a picture, we need to use the \verb=graphicx= package.
In the preamble right after the \verb=\documentclass{..}=,
type \verb=\usepackage{graphicx}=.

Then to include a picture---in EPS, PDF, PNG, or JPG formats---use the command
\verb=\includegraphics{...}=.
For example,
%
%\begin{center}
%    \includegraphics[scale=.5]{hello-yes-this-is-dog}
%\end{center}
%
\begin{framed}\begin{verbatim}
\begin{center}
    \includegraphics[scale=.5]{doggie}
\end{center}\end{verbatim}
\end{framed}
will center the picture \emph{doggie} in its own text block.
The \texttt{scale} option can be used to adjust the size of the image on the page.

\section{Tables}

There are two main table commands:
\begin{itemize}
\item The \verb=tabular= command to arrange figures into rows and columns.
\item The \verb=table= command to produce a floating figure at a convenient piece of the page, say the bottom or the top.
\end{itemize}
Usually the two are used together, with a \verb=\tabular= command nested inside a \verb=\table= wrapper.

The \verb=tabular= environment takes a list of characters giving the justification for each column in the table.
After that, each row of the table is listed. Each row ends with a double backslash (\verb=\\=).
The columns in a row are separated with ampersands (\verb=&=).
For example, the table at the beginning of this document looks like this:

\begin{framed}
\begin{verbatim}
\begin{center}
\begin{tabular}{ll}
    \toprule
    Operating System & Distribution \\
    \midrule
    Windows & MiKTeX \\
    Mac & MacTeX \\
    Linux & TeX Live \\
    \bottomrule
\end{tabular}
\end{center}
\end{verbatim}
\end{framed}

The \verb={ll}= says this table has two columns, and both are left-justified.
The \verb=\toprule=, \verb=\midrule=, \verb=\bottomrule= commands are from the booktabs package.
They draw the lines in the table.
You can leave them out, or if you want to use them add the line \verb=\usepackage{booktabs}= after the \verb=\documentclass= at the beginning of the file.

In the above example, the \verb=tabular= environment has an enclosing \verb=center= environment to center it on the page.
Otherwise, the table will appear in the output file between the same text it appears between in the source file.
The \verb=table= environment lets the table float out of line to the top or bottom of a page.
It provides an area to give a \verb=tabular= environments as well as a caption and a internal reference number (see \S\ref{sec:references}).

It is important to say that it is not necessary to use the \verb=table= environment.
The actual table formatting is done with \verb=tabular=, and the \verb=table= only serves to produce a floating figure.

Example of \verb=table= usage:

\begin{framed}
\begin{verbatim}
\begin{table}[t]
  \centering
  \begin{tabular}
    ... tabular from before ...
  \end{tabular}
  \caption{This table's caption is very modest}
  \label{tbl:modest}
\end{table}
\end{verbatim}
\end{framed}

The \verb=[t]= after the environment says we prefer this table to float to the top of a page.
The \verb=tabular= environment is as before.
A caption is added using \verb=\caption= and the \verb=\label= gives an internal name to this table
for cross-reference purposes (see \S\ref{sec:references}).

\section{Bibliography}

The BibTeX program will generate bibliographies for LaTeX documents.
BibTeX is a separate program which will use an auxiliary file that LaTeX generates to figure out which entries in
your master ``database'' to include in each document's bibliography.
You can manage the BibTeX database either with a program or by hand.
Each citation has an identifying key, which you may assign or change yourself.
To include a citation into a paper use the command \verb=\cite{}=.

For example, if the BibTeX database contains the following entry,
\begin{framed}\begin{verbatim}
@article {black73,
    title = {The pricing of options and corporate liabilities},
    author = {Black, Fischer and Scholes, Myron},
    journal = {The Journal of Political Economy},
    volume = {81},
    issue = {3},
    year = {1973},
    pages = {637--654},
}\end{verbatim}
\end{framed}
which has the key \verb=black73=.
To cite this source in a document write
\begin{framed}\begin{verbatim}
That option was not worth it \cite{black73}.
\end{verbatim}
\end{framed}
which will give
\begin{framed}
    That option was not worth it \cite{black73}.
\end{framed}

To insert the bibliography into your document, put the following code where you
would like it to appear.
\begin{framed}\begin{verbatim}
\bibliographystyle{plain}
\bibliography{research}
\end{verbatim}
\end{framed}

When using citations, up to four passes may be needed to convert your file to a PDF.
\begin{enumerate}
    \item A first pass with LaTeX to generate an \textit{aux} file listing the citations used.
    \item A second pass with BibTeX to generate the bibliography.
    \item A third pass with LaTeX to re-adjust the locations of the sections, tables, etc.
    \item A fourth pass with LaTeX to insert the correct cross-references and table of contents.
\end{enumerate}

%\section{File Organization}
%
%LaTeX generates many auxiliary files when processing a document.
%For this reason, I usually put each document into its own folder.
%Then all the additional files are kept in the same place as the source files,
%and the extra files for different documents don't get mixed together.
%
%For documents larger than an article, such as a book or a dissertation, I have found it helpful to
%make a folder for the overall project, and then put each chapter into its own sub-folder.
%
%\begin{framed}
%\begin{verbatim}
%book-project/
%    references.bib
%    background
%        only.tex
%        background.tex
%    why-spheres
%        only.tex
%        spheres.tex
%    introduction.tex
%    preamble.tex
%    history-of-idea
%        only.tex
%        history.tex
%    book.tex
%\end{verbatim}
%\end{framed}
%
%In the above example directory layout, the file \textit{preamble.tex} contains any shared definitions which I intend to use across the chapters.
%(e.g. I defined the symbol \texttt{\\sss} in there which inserts three large stars so I can see places that I left incomplete.
%\verb=\newcommand{\sss}{\ensuremath{\star\star\star}}=)
%
%The main book is defined in the \textit{book.tex} file.
%It sets up the preamble, title, abstract, table of contents, and then includes each chapter with
%code similar to the following
%
%\begin{framed}
%\begin{verbatim}
%\chapter{Background and Prior Work}
%\label{ch:background}
%\input{background/background}
%\end{verbatim}
%\end{framed}
%
%Inside each chapter folder there are two tex files.
%One holds the main content of the chapter, e.g. \textit{background.tex}.
%This file does not include any preamble stuff nor the \verb=\documentclass= or \verb=\begin{document}= code.
%The other, \textit{only.tex}, wraps the content file so that the chapter can be processed by itself.
%The \textit{only.tex} file will look similar to the following.
%
%\begin{framed}
%\begin{verbatim}
%\documentclass[12pt]{amsart}
%\input{../preamble}
%\begin{document}
%\title{Background and prior work}
%\maketitle
%\input{background}
%\end{document}
%\end{verbatim}
%\end{framed}
%
%To only process a specific chapter, I run LaTeX on the file \textit{background/only.tex}.
%To create the entire book I run LaTeX on the file \textit{book.tex}.

\section{Further Reading}
\label{sec:further-reading}

\begin{itemize}
    \item Andrew Roberts, ``Getting to Grips with LaTeX'', \url{https://www.andy-roberts.net/writing/latex}.

        This is a nice introduction to latex tutorial, similar to this document, but I feel it does a better job of
        surveying all the basic formatting commands.

    \item Wikibooks LaTeX reference, \url{https://en.wikibooks.org/wiki/LaTeX/}

        This is a \textbf{great reference} to keep handy when working on a latex document.
        It covers most common areas of LaTeX in great detail, and offers good extension packages to
        use for special formatting cases.

    \item J\"urgen Fenn, ``Managing Citations and Your Bibliography with BibTeX'', \url{https://www.tug.org/pracjourn/2006-4/fenn/fenn.pdf}

        J\"urgen Fenn article answers many basic questions about using BibTeX and pointers to handling special situations, such as proper names in a title.
        It only shows its age when it discusses the future of BibTeX (section 5.6), where it turns out that BibTeX standard has been accepted by
        libraries and other archives as an export format for citations (but not the ND Library).
        In part the ubiquity of BibTeX is due to the stability of its citation data format.

    \item Michael Downes and Barbara Beeton, ``Short Math Guide for \LaTeX'', \url{http://tug.ctan.org/info/short-math-guide/short-math-guide.pdf}

        This short (21 page) paper lists the extra math commands, symbols, and fonts provided by the American Mathematical Society LaTeX packages
        (\textit{amssymb} and \textit{amsmath}).

    \item Leslie Lamport, ``LATEX: a document preparation system''

        The book by the originator of LaTeX on how to use the system.
        Notre Dame has copies in the library, \url{https://onesearch.library.nd.edu/NDU:malc_blended:ndu_aleph001563762}

\end{itemize}

\section{Other document classes}
Other layouts in addition to \textit{article} are \textit{report}, \textit{letter}, \textit{book}, and \textit{slides}.
Fancier types are the AMS article classes \textit{amsart}, and the slide class \textit{beamer}.


\bibliographystyle{plain}
%\bibliographystyle{alpha}
\bibliography{research}

\end{document}
